\documentclass[11pt,a4paper]{article}
\usepackage[utf8]{inputenc}
\usepackage[margin=2.5cm]{geometry}
\usepackage{hyperref}
\usepackage{enumitem}
\usepackage{xcolor}
\usepackage{tcolorbox}
\usepackage{fontawesome5}

\definecolor{primarycolor}{RGB}{0,102,204}
\definecolor{accentcolor}{RGB}{220,50,50}

\hypersetup{
    colorlinks=true,
    linkcolor=primarycolor,
    urlcolor=primarycolor,
    citecolor=primarycolor
}

\title{\textbf{\Large 5-Day Postdoctoral Technical Challenge}\\[0.3em]
\large AI Medical Imaging + Agentic LLM Pipeline}
\author{}
\date{}

\begin{document}

\maketitle

\begin{tcolorbox}[colback=primarycolor!5,colframe=primarycolor,title=\textbf{Challenge Overview}]
This challenge evaluates your ability to design and implement an end-to-end AI system combining medical imaging, machine learning, and agentic LLM capabilities. You will build a complete pipeline from data processing to automated analysis and reporting.

\vspace{0.5em}
\textbf{Submission:} GitHub repository with complete implementation\\
\textbf{Duration:} 5 days\\
\textbf{Dataset:} MedMNIST v2 -- PneumoniaMNIST (provided)
\end{tcolorbox}

\section{Dataset Specification}

You must use the \textbf{PneumoniaMNIST} dataset from MedMNIST v2:

\begin{itemize}[leftmargin=*]
    \item \textbf{Task:} Binary classification of chest X-ray images for pneumonia detection
    \item \textbf{Size:} Approximately 6,000 training images
    \item \textbf{Format:} 2D grayscale images (28×28 pixels)
    \item \textbf{Access:} Publicly available via Python package
\end{itemize}

\textbf{Installation:}
\begin{verbatim}
pip install medmnist
\end{verbatim}

\textbf{Documentation:} \url{https://medmnist.com/}

\section{Challenge Objectives}

Build a complete system that accomplishes the following:

\begin{enumerate}[leftmargin=*]
    \item Train a deep learning model for pneumonia classification
    \item Evaluate model performance using rigorous metrics
    \item Implement an LLM-based agent to analyze experimental results
    \item Generate an automated experiment report with insights and recommendations
\end{enumerate}

\section{Required Components}

\subsection{1. Data Pipeline}

Implement a complete data processing pipeline including:

\begin{itemize}[leftmargin=*]
    \item Data loading and exploration
    \item Normalization and preprocessing
    \item Data augmentation strategies
    \item Train/validation/test split management
    \item Batch processing for training and inference
\end{itemize}

\subsection{2. Vision Model}

Train a deep learning model for pneumonia classification. You may choose any architecture (CNN, ResNet, Vision Transformer, or custom design).

\textbf{Required outputs:}
\begin{itemize}[leftmargin=*]
    \item Classification accuracy
    \item Area Under the ROC Curve (AUC)
    \item Confusion matrix
    \item Analysis of failure cases with examples
\end{itemize}

\subsection{3. Agentic LLM Component}

Design and implement an LLM-based agent that can analyze experimental results and provide insights.

\textbf{Agent inputs:}
\begin{itemize}[leftmargin=*]
    \item Training metrics and loss curves
    \item Evaluation results and performance statistics
    \item Error analysis data
\end{itemize}

\textbf{Agent outputs:}
\begin{itemize}[leftmargin=*]
    \item Performance explanation and interpretation
    \item Identification of model weaknesses
    \item Concrete suggestions for improvements
    \item Comprehensive experiment summary
\end{itemize}

\textbf{Implementation notes:}
\begin{itemize}[leftmargin=*]
    \item The agent may use tools such as metrics readers, dataset inspectors, or result summarizers
    \item You may use any framework (LangChain, DSPy, custom implementation)
    \item Agent should demonstrate reasoning capabilities beyond simple template filling
\end{itemize}

\subsection{4. Automated Experiment Report}

Your system must automatically generate a comprehensive report including:

\begin{itemize}[leftmargin=*]
    \item Summary of all metrics
    \item Visualization of sample predictions
    \item Error analysis with representative examples
    \item Agent-generated insights and recommendations
    \item Suggested next steps for model improvement
\end{itemize}

\textbf{Output format:} Markdown or PDF

\subsection{5. Reproducibility Requirements}

Your repository must enable complete reproducibility:

\begin{itemize}[leftmargin=*]
    \item Clear README with setup instructions
    \item Complete requirements file with all dependencies
    \item Simple commands to run training and evaluation
    \item Configuration files for hyperparameters
    \item Seed management for deterministic results
\end{itemize}

\section{Expected Repository Structure}

Organize your code with clear separation of concerns:

\begin{verbatim}
repository/
|-- data/              # Data loading and preprocessing
|-- models/            # Model architectures
|-- training/          # Training scripts and utilities
|-- evaluation/        # Evaluation and metrics
|-- agent/             # LLM agent implementation
|-- reports/           # Generated reports and outputs
|-- configs/           # Configuration files
|-- requirements.txt   # Python dependencies
`-- README.md          # Documentation
\end{verbatim}

\section{Evaluation Criteria}

Your submission will be evaluated according to the following rubric:

\begin{center}
\begin{tabular}{|l|c|p{7cm}|}
\hline
\textbf{Area} & \textbf{Weight} & \textbf{Key Aspects} \\
\hline
Pipeline \& Model Quality & 40\% & Code organization, model design, training methodology, performance \\
\hline
Evaluation Rigor & 25\% & Comprehensive metrics, error analysis, statistical validity \\
\hline
Agent Usefulness & 20\% & Quality of insights, reasoning depth, actionable recommendations \\
\hline
Code \& Reproducibility & 15\% & Documentation, code quality, ease of reproduction \\
\hline
\end{tabular}
\end{center}

\section{Bonus Components (Optional)}

Exceptional candidates may include additional features for extra credit:

\begin{itemize}[leftmargin=*]
    \item \textbf{Novel augmentation strategies} tailored to medical imaging
    \item \textbf{Model improvements} through architecture search or ensemble methods
    \item \textbf{Uncertainty estimation} with confidence calibration
    \item \textbf{Training optimization} using advanced techniques
    \item \textbf{Ablation studies} demonstrating component contributions
    \item \textbf{Medical visual embeddings + image retrieval:}
    \begin{itemize}
        \item Use a pre-trained medical vision model (e.g., BioViL-T, MedCLIP, PMC-CLIP)
        \item Build a content-based image retrieval (CBIR) system using FAISS
        \item Implement image-to-image and text-to-image search
        \item Evaluate retrieval quality with Precision@k metrics
        \item Integrate retrieval capabilities with the LLM agent
    \end{itemize}
\end{itemize}

\section{Submission Guidelines}

\subsection{What to Submit}

\begin{enumerate}[leftmargin=*]
    \item \textbf{GitHub repository URL} with complete implementation
    \item \textbf{README.md} with:
    \begin{itemize}
        \item Setup instructions
        \item Commands to reproduce all results
        \item Brief description of your approach
        \item Summary of key findings
    \end{itemize}
    \item \textbf{Generated report} (in reports/ directory)
    \item \textbf{Trained model weights} (or instructions to reproduce)
\end{enumerate}

\subsection{Minimum Viable Submission}

At minimum, your repository must:
\begin{itemize}[leftmargin=*]
    \item Load and preprocess the PneumoniaMNIST dataset
    \item Train a model that achieves reasonable performance (>70\% accuracy)
    \item Generate all required metrics and visualizations
    \item Include a functional LLM agent that produces meaningful analysis
    \item Produce an automated report
    \item Be reproducible with clear documentation
\end{itemize}

\section{Technical Requirements}

\begin{itemize}[leftmargin=*]
    \item \textbf{Programming language:} Python 3.8+
    \item \textbf{Deep learning framework:} PyTorch or TensorFlow
    \item \textbf{LLM access:} You may use any LLM API (OpenAI, Anthropic, open-source models)
    \item \textbf{Version control:} Git with meaningful commit history
    \item \textbf{Documentation:} Clear comments and docstrings
\end{itemize}

\section{Suggested Timeline}

While you may organize your time as you see fit, a typical approach might be:

\begin{itemize}[leftmargin=*]
    \item \textbf{Days 1--2:} Data pipeline and baseline model implementation
    \item \textbf{Day 3:} Comprehensive evaluation and model improvements
    \item \textbf{Day 4:} Agent integration and testing
    \item \textbf{Day 5:} Report generation, documentation, and final cleanup
\end{itemize}

\section{Evaluation Process}

Your submission will be reviewed for:

\begin{enumerate}[leftmargin=*]
    \item \textbf{Technical competence:} Can you build robust ML pipelines?
    \item \textbf{Research thinking:} Do you evaluate thoroughly and think critically?
    \item \textbf{Modern AI integration:} Can you effectively use LLM-based tools?
    \item \textbf{Code quality:} Is your code clean, documented, and maintainable?
    \item \textbf{Problem-solving:} How do you approach challenges and debugging?
\end{enumerate}

\section{Important Notes}

\begin{itemize}[leftmargin=*]
    \item \textbf{Originality:} Your code must be your own work. You may use libraries and reference documentation, but the implementation should demonstrate your understanding.
    \item \textbf{API keys:} If using commercial LLM APIs, ensure you follow best practices for API key management (environment variables, not hardcoded).
    \item \textbf{Computational resources:} The challenge is designed to be completable on a standard laptop. GPU access is helpful but not required.
    \item \textbf{Questions:} If you have clarifying questions about requirements, please reach out.
\end{itemize}

\vspace{1em}
\begin{tcolorbox}[colback=accentcolor!5,colframe=accentcolor]
\textbf{Good luck!} We look forward to reviewing your submission and learning about your approach to this challenge.
\end{tcolorbox}

\end{document}
